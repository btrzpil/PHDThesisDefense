\begin{frame}{Implementacja teleskopowej kaskody ze zintegrowanym sprzężeniem AC}
    \begin{columns}

        \column{.6\textwidth}
        \begin{figure}[H]
            \includegraphics[scale = 0.75]{ch4/chap4Scheme.pdf} 
        \end{figure}

        \column{.35\textwidth}


        \begin{block}{Kluczowe wymagnia}
        \begin{itemize}
            \item optymalizacja szumowa
            \item powierzchnia
            \item pobór mocy
        \end{itemize}
            \end{block}



    \end{columns}   
 \end{frame}



\begin{frame}{Analiza szumowa pary różnicowej}
    \begin{figure}[H]
        \centering
        \includegraphics[scale = 0.75]{scripts/tmp/differentialPair.pdf}  
    \end{figure}

\end{frame}



\begin{frame}{}
    \begin{columns}

    \column{.65\textwidth}
    \begin{figure}[H]
        \centering
        \includegraphics[scale=0.5]{ch4/channel.pdf} 
    \end{figure}   

    \column{.35\textwidth}

    \begin{block}{
        Przedwzmacniacz z wejściowym obwodem sprzęgającym AC oraz obwody polaryzujące
    }
    {\renewcommand\normalsize{\small}%
    \normalsize
    \begin{table}[H]
        \centering
        \begin{tabular}{lll} 
        \toprule
        \begin{tabular}[c]{@{}l@{}}Kluczowe \\tranzystory\end{tabular} & $W$ [$\SI{}{\micro\metre}$] & $L$ [$\SI{}{\micro\metre}$]  \\ 
        \toprule
        $M_{bias}$                                                       & 10                          & 10                           \\
        $M_1,\ M_2$                                                    & 300                         & 1                            \\
        $M_3,\ M_4$                                                    & 20                          & 2                            \\
        $M_5,\ M_6$                                                    & 5                           & 5                            \\
        $M_7,\ M_8$                                                    & 4                           & 48                           \\
        \bottomrule
        \end{tabular}
    \end{table}
    }
    \end{block}
    \end{columns}   
  
\end{frame}


\begin{frame}{Blok korekcji}
\begin{columns}

    \column{.35\textwidth}
    \begin{block}{
        Projekt kanału}
        \begin{figure}[H]
            \centering
            \includegraphics[scale = 0.4]{ch4/chap4Scheme.pdf}
        \end{figure} 
        \end{block}

        \begin{block}{
            Wyzwania do rozwiązania}
            \begin{figure}[H]
                \centering
                \includegraphics[scale = 0.6]{ch4/vgs_corr_sch.pdf} 
            \end{figure}   
            \end{block}



    \column{.6\textwidth}
    \begin{columns}
    \column{.45\textwidth}

    \begin{figure}[H]
        \centering
        \includegraphics[scale = 0.45]{scripts/tmp/analyseVgsTHD_1.pdf}
    \end{figure} 
    \column{.45\textwidth}
    \begin{figure}[H]
        \centering
        \includegraphics[scale =0.45]{scripts/tmp/analyseVgsTHD_2.pdf}
    \end{figure} 
    \end{columns}   

    \begin{columns}
    \column{.45\textwidth}

    \begin{figure}[H]
        \centering
        \includegraphics[scale = 0.4]{ch4/vgs_corr0.pdf}
    \end{figure} 
    \column{.45\textwidth}
    \begin{figure}[H]
        \centering
        \includegraphics[scale = 0.4]{ch4/vgs_corr1.pdf}
    \end{figure} 
    \end{columns}   
\end{columns}  
\end{frame}


\begin{frame}{}
    \begin{columns}

    \column{.45\textwidth}
    \begin{block}{}
        \begin{itemize}
            \item 8 wersji przedwzmacniacza
            \item 14 kanałów pomiarowych
            \item cztery wersje rozmiarów tranzystorów PMOS tworzących pseudo-rezystory  -- odpowiednio $W/L$: $2/40,\ 1/40,\ 2/20,\ 1/20\ \SI{}{\micro\metre / \micro\metre}$
            \item 2 konfiguracje pojemności -- $C_{in}/C_f = 4/200\ \SI{}{\pico\farad}/\SI{}{\femto\farad}$
        \end{itemize}
    \end{block}

\vspace{-2em}
    \begin{figure}[H]
        \centering
        \includegraphics[trim={0 12cm 0 0},clip, scale = 0.5]{ch4/layoutASIC.pdf} 
    \end{figure}   
    \column{.5\textwidth}

    \begin{block}{
Symulacje Post-Layout
    }

    \begin{figure}[H]
        \centering
        \includegraphics[scale = 0.5]{scripts/tranSchematicLayout/tranSchematicLayout.pdf}  
    \end{figure}
    \vspace{-5mm} %5mm vertical space
    \begin{figure}[H]
        \centering
        \includegraphics[scale = 0.5]{scripts/noiseContribution/noiseContributionOut.pdf}  
    \end{figure}
    \end{block}
    \end{columns}   
  
\end{frame}



