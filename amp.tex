% \begin{frame}{Implementacja teleskopowej kaskody ze zintegrowanym sprzężeniem AC}
%     \begin{columns}

%         \column{.6\textwidth}
%         \begin{figure}[H]
%             \includegraphics[scale = 0.75]{ch4/chap4Scheme.pdf} 
%         \end{figure}

%         \column{.35\textwidth}


%         \begin{block}{Kluczowe wymagnia}
%         \begin{itemize}
%             \item optymalizacja szumowa
%             \item powierzchnia
%             \item pobór mocy
%         \end{itemize}
%             \end{block}



%     \end{columns}   
%  \end{frame}



% \begin{frame}{Analiza szumowa pary różnicowej}
%     \begin{figure}[H]
%         \centering
%         \includegraphics[scale = 0.75]{scripts/tmp/differentialPair.pdf}  
%     \end{figure}

% \end{frame}



\begin{frame}{Przedwzmacniacz z wejściowym obwodem sprzęgającym AC}
    \begin{columns}

    \column{.65\textwidth}
    \begin{figure}[H]
        \centering
        \includegraphics[scale=0.45]{ch4/channel.pdf} 
    \end{figure}   

    \column{.35\textwidth}

    \begin{block}{
    }
    {\renewcommand\normalsize{\small}%
    \normalsize
    \begin{itemize}
        \item Konfiguracja  teleskopowej kaskody  jako aktywny OTA
        \item Polaryzacja pary różnicowej w obszarze podprogowym pracy tranzystora
        \item  Napięcie zasilania $\SI{\pm 1.8}{\volt}$
    \end{itemize}
    \vspace{-1em}
    \begin{table}[H]
        \centering
        \begin{tabular}{lll} 
        \toprule
        \begin{tabular}[c]{@{}l@{}}Kluczowe \\tranzystory\end{tabular} & $W$ [$\SI{}{\micro\metre}$] & $L$ [$\SI{}{\micro\metre}$]  \\ 
        \toprule
        $M_{bias}$                                                       & 10                          & 10                           \\
        $M_1,\ M_2$                                                    & 300                         & 1                            \\
        $M_3,\ M_4$                                                    & 20                          & 2                            \\
        $M_5,\ M_6$                                                    & 5                           & 5                            \\
        $M_7,\ M_8$                                                    & 4                           & 48                           \\
        \bottomrule
        \end{tabular}
    \end{table}
    }
    \end{block}
    \end{columns}   
  
\end{frame}


% \begin{frame}{Blok korekcji}
% \begin{columns}

%     \column{.35\textwidth}
%     \begin{block}{
%         Projekt kanału}
%         \begin{figure}[H]
%             \centering
%             \includegraphics[scale = 0.4]{ch4/chap4Scheme.pdf}
%         \end{figure} 
%         \end{block}

%         \begin{block}{
%             Wyzwania do rozwiązania}
%             \begin{figure}[H]
%                 \centering
%                 \includegraphics[scale = 0.6]{ch4/vgs_corr_sch.pdf} 
%             \end{figure}   
%             \end{block}



%     \column{.6\textwidth}
%     \begin{columns}
%     \column{.45\textwidth}

%     \begin{figure}[H]
%         \centering
%         \includegraphics[scale = 0.45]{scripts/tmp/analyseVgsTHD_1.pdf}
%     \end{figure} 
%     \column{.45\textwidth}
%     \begin{figure}[H]
%         \centering
%         \includegraphics[scale =0.45]{scripts/tmp/analyseVgsTHD_2.pdf}
%     \end{figure} 
%     \end{columns}   

%     \begin{columns}
%     \column{.45\textwidth}

%     \begin{figure}[H]
%         \centering
%         \includegraphics[scale = 0.4]{ch4/vgs_corr0.pdf}
%     \end{figure} 
%     \column{.45\textwidth}
%     \begin{figure}[H]
%         \centering
%         \includegraphics[scale = 0.4]{ch4/vgs_corr1.pdf}
%     \end{figure} 
%     \end{columns}   
% \end{columns}  
% \end{frame}


\begin{frame}{}
    \begin{columns}

    \column{.45\textwidth}
    \begin{block}{}
        {\renewcommand\normalsize{\small}%
        \normalsize
        \begin{itemize}
            \item 8 wersji przedwzmacniacza i 14 kanałów
            \item 4 wersje tranzystorów PMOS tworzących pseudo-rezystory  --  $W/L$: $2/40,\ 1/40,\ 2/20,\ 1/20\ \SI{}{\micro\metre / \micro\metre}$
            \item 2 konfiguracje pojemności -- $C_{in}/C_f = 4/200,\ 8/400\ \SI{}{\pico\farad}/\SI{}{\femto\farad}$
        \end{itemize}
        }
    \end{block}

\vspace{-1em}
    \begin{figure}[H]
        \centering
        \includegraphics[trim={0 12cm 0 0},clip, scale = 0.5]{ch4/layoutASIC.pdf} 
    \end{figure}   
    \column{.5\textwidth}

    \begin{block}{
Symulacje Post-Layout
    }

    \begin{figure}[H]
        \centering
        \includegraphics[scale = 0.45]{scripts/tranSchematicLayout/tranSchematicLayout.pdf}  
    \end{figure}
    \vspace{-5mm} %5mm vertical space
    \begin{figure}[H]
        \centering
        \includegraphics[scale = 0.45]{scripts/noiseContribution/noiseContributionOut.pdf}  
    \end{figure}
    \end{block}
    \end{columns}   
  
\end{frame}



