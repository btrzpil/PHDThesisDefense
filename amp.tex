\begin{frame}{Implementacja teleskopowej kaskody ze zintegrowanym sprzężeniem AC}
    \begin{columns}

        \column{.6\textwidth}
        \begin{figure}[H]
            \includegraphics[scale = 0.8]{ch4/chap4Scheme.pdf} 
        \end{figure}

        \column{.35\textwidth}

        \begin{figure}[H]
            \centering
            \includegraphics[scale = 0.5]{scripts/tmp/differentialPair.pdf}  
        \end{figure}

        \vspace{-5mm} %5mm vertical space
        {\renewcommand\normalsize{\small}%
\normalsize
\begin{table}[H]
    \centering
    \begin{tabular}{lll} 
    \toprule
    \begin{tabular}[c]{@{}l@{}}Kluczowe \\tranzystory\end{tabular} & $W$ [$\SI{}{\micro\metre}$] & $L$ [$\SI{}{\micro\metre}$]  \\ 
    \toprule
    $M_{bias}$                                                       & 10                          & 10                           \\
    $M_1,\ M_2$                                                    & 300                         & 1                            \\
    $M_3,\ M_4$                                                    & 20                          & 2                            \\
    $M_5,\ M_6$                                                    & 5                           & 5                            \\
    $M_7,\ M_8$                                                    & 4                           & 48                           \\
    \bottomrule
    \end{tabular}
\end{table}
}


    \end{columns}   
 \end{frame}







\begin{frame}{}
    \begin{columns}

    \column{.75\textwidth}
    \begin{figure}[H]
        \centering
        \includegraphics[scale=0.5]{ch4/channel.pdf} 
    \end{figure}   

    \column{.2\textwidth}

    \begin{block}{
        Przedwzmacniacz z wejściowym obwodem sprzęgającym AC oraz obwody polaryzujące
    }

    \end{block}
    \end{columns}   
  
\end{frame}


\begin{frame}{Efekty niedopasowania -- blok korekcji}
\begin{columns}

    \column{.35\textwidth}
    \begin{block}{
        Projekt kanału}
        \begin{figure}[H]
            \centering
            \includegraphics[scale = 0.4]{ch4/chap4Scheme.pdf}
        \end{figure} 
        \end{block}




    \column{.6\textwidth}
    \begin{columns}
    \column{.45\textwidth}

    \begin{figure}[H]
        \centering
        \includegraphics[scale = 0.5]{scripts/tmp/analyseVgsTHD_1.pdf}
    \end{figure} 
    \column{.45\textwidth}
    \begin{figure}[H]
        \centering
        \includegraphics[scale =0.5]{scripts/tmp/analyseVgsTHD_2.pdf}
    \end{figure} 
    \end{columns}   
\end{columns}  

\begin{columns}

\column{.35\textwidth}
\begin{block}{
Wyzwania do rozwiązania}
\begin{figure}[H]
    \centering
    \includegraphics[scale = 0.6]{ch4/vgs_corr_sch.pdf} 
\end{figure}   
\end{block}

\column{.6\textwidth}
    \begin{columns}
    \column{.45\textwidth}

    \begin{figure}[H]
        \centering
        \includegraphics[scale = 0.5]{ch4/vgs_corr0.pdf}
    \end{figure} 
    \column{.45\textwidth}
    \begin{figure}[H]
        \centering
        \includegraphics[scale =0.5]{ch4/vgs_corr1.pdf}
    \end{figure} 
    \end{columns}   
\end{columns}  
\end{frame}


\begin{frame}{}
    \begin{columns}

    \column{.35\textwidth}
    TODo - usunąć dół, dac tabelke z wersjami

    \begin{figure}[H]
        \centering
        \includegraphics[scale=0.5]{ch4/layoutASIC.pdf} 
    \end{figure}   
    \column{.6\textwidth}

    \begin{block}{
        Symulacje z uwzględnieniem elementów pasożytniczych
    }

    \begin{figure}[H]
        \centering
        \includegraphics[scale = 0.6]{scripts/tranSchematicLayout/tranSchematicLayout.pdf}  
    \end{figure}

    \begin{figure}[H]
        \centering
        \includegraphics[scale = 0.6]{scripts/noiseContribution/noiseContributionOut.pdf}  
    \end{figure}
    \end{block}
    \end{columns}   
  
\end{frame}



