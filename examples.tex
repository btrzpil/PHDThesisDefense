% \ifdefined\textleftmargin
% 	%%%%%%%%%%%%%%%%
% 	\begin{frame}[fragile]{\iflanguage{polish}{Informacje}{Information}}
% 		\begin{center}
% 			\iflanguage{polish}{
% 				$\Longleftarrow$ Aktualna wartość \hfill Aktualna wartość $\Longrightarrow$\\
% 				$\Longleftarrow$ rozmiaru lewego marginesu \hfill rozmiaru prawego marginesu $\Longrightarrow$\\
% 				$\Longleftarrow$ to \the\textleftmargin \hfill to \the\textrightmargin $\Longrightarrow$
% 			}{
% 				$\Longleftarrow$ The current value of \hfill The current value of $\Longrightarrow$\\
% 				$\Longleftarrow$ the  left  margin size \hfill the right margin size $\Longrightarrow$\\
% 				$\Longleftarrow$ is \the\textleftmargin \hfill is \the\textrightmargin $\Longrightarrow$
% 			}
% 		\end{center}
% 		\iflanguage{polish}{
% 			Możesz je zmieniać za pomocą parametru 'margins' \pauza
% 		}{
% 			You can change them with the 'margins' parameter ---
% 		}
% 		\verb+\usetheme[margins=...]{AGH}+
% 	\end{frame}
% \fi
%%%%%%%%%%%%%%%%

%%%%%%%%%%%%%%%%
\begin{frame}{\iflanguage{polish}{Plan prezentacji}{Outline}}
	\tableofcontents[pausesections]
\end{frame}
%%%%%%%%%%%%%%%%
\section{\iflanguage{polish}{Elementy podstawowe}{Basic elements}}
%%%%%%%%%%%%%%%%
\begin{frame}{\iflanguage{polish}{Wyszczególnienie}{Itemize}}
	\begin{columns}
		\column{0.5\textwidth}
		\begin{itemize}
			\item \iflanguage{polish}{Element 1}{Item 1}
			\item \iflanguage{polish}{Element 2}{Item 2}
			\item \iflanguage{polish}{Element 3}{Item 3}
		\end{itemize}
		\column{0.5\textwidth}
		\pause
		\structure{\iflanguage{polish}{Odkrywanie po kolei}{Uncovering one by one}}
		\begin{itemize}[<+->]
			\item \iflanguage{polish}{Element 1}{Item 1}
			\item \iflanguage{polish}{Element 2}{Item 2}
			\item \iflanguage{polish}{Element 3}{Item 3}
		\end{itemize}
		\onslide
	\end{columns}
\end{frame}
%%%%%%%%%%%%%%%%
\begin{frame}{\iflanguage{polish}{Wyliczenie}{Enumerate}}
	\begin{columns}
		\column{0.5\textwidth}
		\begin{enumerate}
			\item \iflanguage{polish}{Element 1}{Item 1}
			\item \iflanguage{polish}{Element 2}{Item 2}
			\item \iflanguage{polish}{Element 3}{Item 3}
		\end{enumerate}
		\column{0.5\textwidth}
		\pause
		\structure{\iflanguage{polish}{Odkrywanie elementów po kolei z jednoczesnym wyróżnianiem}{Uncovering elements in turn with simultaneous highlighting}}
		\begin{enumerate}[<+-|alert@+>]
			\item \iflanguage{polish}{Element 1}{Item 1}
			\item \iflanguage{polish}{Element 2}{Item 2}
			\item \iflanguage{polish}{Element 3}{Item 3}
		\end{enumerate}
		\onslide
	\end{columns}
\end{frame}
%%%%%%%%%%%%%%%%
\section{\iflanguage{polish}{Matematyka}{Mathematics}}
%%%%%%%%%%%%%%%%
\begin{frame}{\iflanguage{polish}{Podstawowe bloki}{Basic blocks}}
	% Examples from "The beamer class User Guide"
	\iflanguage{polish}{
		\begin{block}{Definicja}
			\alert{Zbiór} składa się z elementów.
		\end{block}
		\begin{exampleblock}{Przykład}
			Zbiór $\{1,2,3,5\}$ zawiera cztery elementy.
		\end{exampleblock}
		\begin{alertblock}{Błędne Twierdzenie}
			$1=2$.
		\end{alertblock}
	}{
		\begin{block}{Definition}
			A \alert{set} consists of elements.
		\end{block}
		\begin{exampleblock}{Example}
			The set $\{1,2,3,5\}$ has four elements.
		\end{exampleblock}
		\begin{alertblock}{Wrong Theorem}
			$1=2$.
		\end{alertblock}
	}
\end{frame}
%%%%%%%%%%%%%%%%
\begin{frame}{\iflanguage{polish}{Otoczenia matematyczne}{Math environments}}
	\begin{columns}
		\column{0.45\textwidth} %The first column
		\structure{\iflanguage{polish}{Twierdzenia}{Theorems}}
		\begin{theorem}[\iflanguage{polish}{Pitagorasa}{Pythagorean}]
			$a^{2}+  b^{2}=  c^{2}$
		\end{theorem}
		\column{0.45\textwidth} %The second column
		\structure{\iflanguage{polish}{Dowody}{Proofs}}
		\begin{proof}
			\ldots
		\end{proof}
	\end{columns}
	\vfill
	\ldots
	\vfill
	\begin{definition}
		\ldots
	\end{definition}
\end{frame}
%%%%%%%%%%%%%%%%
\begin{frame}{\iflanguage{polish}{Dynamiczny wzór matematyczny}{Dynamic mathematical formula}}
	\[
		\binom{n}{k} = \pause \frac{n!}{k!(n-k)!}
	\]
\end{frame}
%%%%%%%%%%%%%%%%
% \section{\iflanguage{polish}{Informatyka}{Computer Science}}
% %%%%%%%%%%%%%%%%
% \subsection*{\iflanguage{polish}{Wstawianie kodów źródłowych}{Inserting source codes}}
% %%%%%%%%%%%%%%%%
% \begin{frame}[fragile]{\iflanguage{polish}{Użycie otoczenia 'listings'}{Using the 'listings' environment}}
% 	\begin{lstlisting}[language=C++]
% /* The first  program in C++ */  %*\pause*)
% #include <iostream>  %*\pause*)
% using namespace std; %*\pause*)
% void main() 
% {       %*\pause*)
%   cout %*\pause*) << "Hello World!"%*\pause*) << endl; %*\onslide<4->*)
% } %*\onslide*)
% \end{lstlisting}
% \end{frame}
%%%%%%%%%%%%%%%%
% \begin{frame}[fragile]{\iflanguage{polish}{Użycie otoczenia 'minted'}{Using the 'minted' environment}}
% 	\begin{minted}[beameroverlays,escapeinside=||]{C++}
% /* The first  program in C++ */  |\pause|
% #include <iostream>  |\pause|
% using namespace std; |\pause|
% void main() 
% {       |\pause|
%   cout |\pause| << "Hello World!"|\pause| << endl; |\onslide<4->|
% } |\onslide|
% 	\end{minted}
% \end{frame}
% %%%%%%%%%%%%%%%%%%%%%%%
% \appendix
% %%%%%%%%%%%%%%%%%%%%%%%
% \begin{frame}[allowframebreaks]{\iflanguage{polish}{Bibliografia}{Bibliography}}
% 	\begin{thebibliography}{9}
% 		\setbeamertemplate{bibliography item}[online]
% 		\bibitem{wikibook}{Wikibooks \newblock \LaTeX/Source Code Listings \newblock \url{https://en.wikibooks.org/wiki/LaTeX/Source_Code_Listings}}
% 		\bibitem{beamer}{Till Tantau, Joseph Wright, Vedran Miletić \newblock The beamer class \newblock \url{http://mirror.ctan.org/macros/latex/contrib/beamer/doc/beameruserguide.pdf}}
% 		\setbeamertemplate{bibliography item}[book]
% 		\bibitem{lamport}{Leslie Lamport \newblock LATEX: a document preparation system : user's guide and reference manual \newblock Addison-Wesley Pub. Co., 1994 }
% 		\setbeamertemplate{bibliography item}[article]
% 		\iflanguage{polish}{
% 			\bibitem{article1}{Autor \newblock Tytuł artykułu \newblock Edytor, rok \newblock Uwagi}
% 			\setbeamertemplate{bibliography item}[triangle]
% 			\bibitem{article2}{Autor \newblock Tytuł artykułu \newblock Edytor, rok \newblock Uwagi}
% 			\setbeamertemplate{bibliography item}[text]
% 			\bibitem{article3}{Autor \newblock Tytuł artykułu \newblock Edytor, rok \newblock Uwagi}
% 			\bibitem[Polak98]{article4}{Autor \newblock Tytuł artykułu \newblock Edytor, rok \newblock Uwagi}
% 		}{
% 			\bibitem{article1}{Author \newblock Title of the article\newblock Editor, year \newblock Notes}
% 			\setbeamertemplate{bibliography item}[triangle]
% 			\bibitem{article2}{Author \newblock Title of the article\newblock Editor, year \newblock Notes}
% 			\setbeamertemplate{bibliography item}[text]
% 			\bibitem{article3}{Author \newblock Title of the article\newblock Editor, year \newblock Notes}
% 			\bibitem[Polak98]{article4}{Author \newblock Title of the article\newblock Editor, year \newblock Notes}
% 		}
% 	\end{thebibliography}
% \end{frame}