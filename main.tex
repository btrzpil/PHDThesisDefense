\documentclass[aspectratio=1610, polish]{beamer} 
% Jeśli chcesz otrzymać prezentację w języku polskim, to, powyżej, zamień „english” na „polish”
\usepackage{babel}
\makeatletter
\@ifclasswith{beamer}{polish}{
	\usepackage{polski}
}
\makeatother
\usepackage[utf8]{inputenc}
\usepackage{listings} % We want to put listings
% \usepackage{minted}   % We want to put listings

\mode<beamer>{ 	% In the 'beamer' mode
	\hypersetup{pdfpagemode=FullScreen}         % Enable Full screen mode
	\usetheme[parttitle=rightfooter, nosidebar, margins=1em]{AGH}       % Show part title in right footer
	%\usetheme[nosidebar]{AGH}                  % Do not show sidebar on non-title slides
	% \usetheme[nosidebar, margins=1em]{AGH}     % Do not show sidebar on non-title slides and set both margins (left / right) to 1em
	%\usetheme[dark]{AGH}                       % Use dark background
	%\usetheme[dark, parttitle=leftfooter]{AGH} % Use dark background and show part title in left footer
}
\mode<handout>{	% In the 'handout' mode
	\hypersetup{pdfpagemode=None}		
	\usepackage{pgfpages}
	\pgfpagesuselayout{4 on 1}[a4paper,border shrink=5mm,landscape]	% Show 4 slides on 1 page
	\pgfpageslogicalpageoptions{1}{border code=\pgfusepath{stroke}}
	\pgfpageslogicalpageoptions{2}{border code=\pgfusepath{stroke}}
	\pgfpageslogicalpageoptions{3}{border code=\pgfusepath{stroke}}
	\pgfpageslogicalpageoptions{4}{border code=\pgfusepath{stroke}}
  	\usetheme{boxes}
  	\addheadbox{structure}{\quad\insertpart\hfill\insertsection\hfill\insertsubsection\qquad}          % Content of header
 	\addfootbox{structure}{\quad\insertshortauthor\hfill\insertframenumber\hfill\insertsubtitle\qquad} % Content of footer
}

\AtBeginPart{ % At begin part: display its name
	\frame{\partpage}
} 
\author[Beata Trzpil-Jurgielewicz]{mgr inż. Beata \textsc{Trzpil-Jurgielewicz}
\newline 
promotorzy:
\newline
prof. dr hab. inż. Władysław \textsc{Dąbrowski} 
\newline 
dr inż. Paweł \textsc{Hottowy}}
\date{}
\iflanguage{polish}
{
	\title[]{Opracowanie wielokanałowego układu scalonego w technologii CMOS do rejestracji aktywności neuronalnej oraz jego aplikacja w funkcjonalnych badaniach mózgu
	}
	% \institute[AGH]{
	% 		\inst{1}Instytut Informatyki\newline
	% 		ul. Kawiory 21\newline
	% 		30-055 Kraków\newline
	% 		\url{http://www.icsr.agh.edu.pl/~polak/}
	% 	\and
	% 		\inst{2}Druga afiliacja
 	% }
}{
	\title{---}
	% \institute[AGH]{
	% 		\inst{1}Institute of Computer Science\newline
	% 		Kawiory 21 Street\newline
	% 		30-055 Kraków\newline
	% 		Poland\newline
	% 		\url{http://www.icsr.agh.edu.pl/~polak/}
	% 	\and
	% 		\inst{2}Second affiliation
	% }
}
%%%%%%%%%%% Configuration of the listings package %%%%%%%%%%%%%%%%%%%%%%%%%%
% Source: https://en.wikibooks.org/wiki/LaTeX/Source_Code_Listings#Using_the_listings_package
%%%%%%%%%%%%%%%%%%%%%%%%%%%%%%%%%%%%%%%%%%%%%%%%%%%%%%%%%%%%%%%%%%%%%%%%%%%%
\lstset{ %
  backgroundcolor=\color{white},   % choose the background color
  basicstyle=\footnotesize,        % the size of the fonts that are used for the code
  breakatwhitespace=false,         % sets if automatic breaks should only happen at whitespace
  breaklines=true,                 % sets automatic line breaking
  captionpos=b,                    % sets the caption-position to bottom
  commentstyle=\color{green},      % comment style
  deletekeywords={...},            % if you want to delete keywords from the given language
  escapeinside={\%*}{*)},          % if you want to add LaTeX within your code
  extendedchars=true,              % lets you use non-ASCII characters; for 8-bits encodings only, does not work with UTF-8
  frame=single,	                   % adds a frame around the code
  keepspaces=true,                 % keeps spaces in text, useful for keeping indentation of code (possibly needs columns=flexible)
  keywordstyle=\color{blue},       % keyword style
  morekeywords={*,...},            % if you want to add more keywords to the set
  numbers=left,                    % where to put the line-numbers; possible values are (none, left, right)
  numbersep=5pt,                   % how far the line-numbers are from the code
  numberstyle=\tiny\color{gray},   % the style that is used for the line-numbers
  rulecolor=\color{black},         % if not set, the frame-color may be changed on line-breaks within not-black text (e.g. comments (green here))
  showspaces=false,                % show spaces everywhere adding particular underscores; it overrides 'showstringspaces'
  showstringspaces=false,          % underline spaces within strings only
  showtabs=false,                  % show tabs within strings adding particular underscores
  stepnumber=2,                    % the step between two line-numbers. If it's 1, each line will be numbered
  stringstyle=\color{cyan},        % string literal style
  tabsize=2,	                   % sets default tabsize to 2 spaces
  title=\lstname,                  % show the filename of files included with \lstinputlisting; also try caption instead of title
                                   % needed if you want to use UTF-8 Polish chars
  literate={ą}{{\k{a}}}1
           {Ą}{{\k{A}}}1
           {ę}{{\k{e}}}1
           {Ę}{{\k{E}}}1
           {ó}{{\'o}}1
           {Ó}{{\'O}}1
           {ś}{{\'s}}1
           {Ś}{{\'S}}1
           {ł}{{\l{}}}1
           {Ł}{{\L{}}}1
           {ż}{{\.z}}1
           {Ż}{{\.Z}}1
           {ź}{{\'z}}1
           {Ź}{{\'Z}}1
           {ć}{{\'c}}1
           {Ć}{{\'C}}1
           {ń}{{\'n}}1
           {Ń}{{\'N}}1
}
\usepackage{booktabs}
\usepackage{siunitx}
\sisetup{locale = PL,
         exponent-product=\ensuremath{\times}}
\graphicspath{{../PhD_Thesis_BTJ/Figures/}}
% \graphicspath{{../BTJ_Thesis/Figures/}}

\usepackage{caption}
\usepackage{subcaption}
\usepackage{tabularray}

\usepackage[most]{tcolorbox}    	% for COLORED BOXES (tikz and xcolor included)
\usepackage{ninecolors}
\selectcolormodel{rgb}
\definecolor{main}{HTML}{1F77B4}    % setting main color to be used
\definecolor{sub}{HTML}{cde4ff}     % setting sub color to be used
\definecolor{deepblue}{HTML}{1F77B4}
\definecolor{deepred}{HTML}{D92728}
\definecolor{lightred}{HTML}{D98789}
\definecolor{deepgreen}{HTML}{2CA02C}

\tcbset{
    sharp corners,
    colback = white,
    before skip = 0.2cm,    % add extra space before the box
    after skip = 0.5cm      % add extra space after the box
}  

\newtcolorbox{boxR}{
    enhanced, % for a fancier setting,
    boxrule = 0pt, % clearing the default rule
    borderline = {0.75pt}{0pt}{deepred}, % outer line
    borderline = {0.75pt}{2pt}{lightred}, % inner line
    % title = sda
}

%%%%%%%%%%% Configuration of the minted package %%%%%%%%%%%%%%%%%%%%%%%%%%
% \setminted[C++]{frame=single,linenos}

%%%%%%%%%%%%%%%%%
\begin{document}
\maketitle


% \part{Tematyka pracy}
\begin{frame}{Plan prezentacji}
	\tableofcontents%[pausesections]
\end{frame}
\section{Systemy do rejestracji aktywności elektrycznej żywych tkanek nerwowych}
\section{Projekt liniowego pseudo-rezystora w zakresie \si{\giga\ohm}}
\section{Operacyjny wzmacniacz transkonduktancyjny}
\section{Weryfikacja elektroniczna i neurofizjologiczna układu scalonego HiFiNeuroPre}

\part{Tematyka pracy}
% \begin{frame}{}
%     \begin{columns}
%         \column{.48\textwidth}


%         \column{.48\textwidth}
%         \begin{figure}[H]
%             \centering
%             \includegraphics[scale=1.0]{ch1/lfp_ap_spectrum}  
%             \end{figure}	
%     \end{columns}


\begin{frame}{Zakresy amplitud i częstotliwości sygnałów neuronowych}
    \begin{columns}
        \column{.48\textwidth}
        \begin{figure}[H]
            \includegraphics[scale=0.2]{ch1/brain.jpg}
          \end{figure}

        \column{.48\textwidth}
        \begin{figure}[H]
            \centering
            \includegraphics[scale=1.0]{ch1/lfp_ap_spectrum}  
            \end{figure}	
    \end{columns}
    
\end{frame}



\begin{frame}{Schemat typowego kanału rejestracji neuronowej z wykorzystaniem elektrod zewnątrzkomórkowych}

    \begin{columns}
        \column{.48\textwidth}
        \begin{figure}[H]
            \centering
            \includegraphics[scale=0.25]{ch2/chemNeuroInterface.png} 
        \end{figure}
        \column{.48\textwidth}
        % Schemat typowego kanału rejestracji neuronowej i modelu elektrycznego interfejsu tkanka-mikroelektroda: 
        % $Z_{CPA}$ -- element o stałej fazie, 
        % $R_{CT}$ -- rezystancja dla przepływającego prądu przez elektrodę,  
        % $R_{SP}$ -- rezystancja rozproszona elektrolitu, 
        % $V_{HC}$ -- potencjał w interfejsie elektroda -- tkanka. 
 
    \end{columns}
\end{frame}

\begin{frame}{Wymagania stawiane interfejsom neuroelektronicznym umożliwiającym rejestrację sygnałów LFP i AP}

    
\end{frame}

\begin{frame}{Sprzężenie zmiennoprądowe}
    \begin{columns}
        \column{.48\textwidth}

    \begin{figure}[H]
        \centering
        \includegraphics[scale=1.0]{ch2/conceptAC_Harrison.pdf} 
    \end{figure}
    \column{.48\textwidth}
    \begin{alertblock}{Wyzwania zwiazane z sprzęzeniem AC}
pojemności rezystancja wzmocnienie
    \end{alertblock}
    \begin{exampleblock}{Zalety}

    \end{exampleblock}
\end{columns}

\end{frame}



\begin{frame}{Sprzężenie stałoprądowe}
    \begin{columns}
        \column{.48\textwidth}
        \begin{figure}[H]
            \centering
            \includegraphics[scale = 0.7]{ch2/dc_coupling.pdf}
        \end{figure}

    \column{.48\textwidth}
    \begin{alertblock}{Wyzwania zwiazane z sprzęzeniem DC}
saturacja
    \end{alertblock}
    \begin{exampleblock}{Zalety}

    \end{exampleblock}
\end{columns}

\end{frame}




\part{Liniowy pseudo-rezystor}


\begin{frame}{Podstawowe rozwiązania psuedo-rezystorów}


        \begin{figure}[H]
            \centering
            \includegraphics[scale = 0.6]{ch2/ntpr.pdf}

        \end{figure}
        \vspace{-5mm} %5mm vertical space
        % Różne topologie pseudo-rezystorów z niekontrolowaną wartością rezystancji zaimplementowane w przedwzmacniaczu ze sprzężeniem zmiennoprądowym.        
        \begin{alertblock}{Wady}
            \begin{itemize}
                \item stała rezystancja 
                \item brak możliwości regulacji częstotliwości granicznej
            \end{itemize}
        \end{alertblock}



\end{frame}

\begin{frame}{Podstawowe rozwiązania psuedo-rezystorów - regulowana wartość rezystancji}
    \begin{block}{}
        Regulacja częstotliowści granicznej
    \end{block}

    \begin{columns}
        \column{.58\textwidth}
        \hspace{-10mm} %5mm vertical space
        \begin{figure}[H]

            \includegraphics[scale = 0.6]{ch2/tpr.pdf}

        \end{figure}
        \column{.35\textwidth}
        % Różne topologie pseudo-rezystorów z kontrolowaną wartością rezystancji zaimplementowane w przedwzmacniaczu ze sprzężeniem zmiennoprądowym.
        \vspace{-5mm} \begin{alertblock}{}
        Zmieniające się napięcie panujące pomiędzy bramką a źródłem tranzystora w zależności od sygnału wejściowego
        \end{alertblock}
        \vspace{5mm} 
        \begin{exampleblock}{}
            Zachowanie stałego napięcia pomiędzy bramką, a źródłem niezależnie od sygnału wejsciowego 

        \end{exampleblock}
    \end{columns}


\end{frame}







\begin{frame}{Analiza stałoprądowa}

        
    \vspace{-1em}


    \begin{columns}
        \column{.48\textwidth}
         \begin{alertblock}{}
            \begin{figure}[H]
                \includegraphics[scale=0.8]{ch3/ptune.pdf}
            \end{figure}
            \end{alertblock}
        \column{.48\textwidth}
        \begin{exampleblock}{}
            \begin{figure}[H]
                \includegraphics[scale=0.8]{ch3/pvgs.pdf}
            \end{figure}

        \end{exampleblock}
    \end{columns}

\vspace{-0.5em}
    \begin{columns}
        \column{.48\textwidth}
        \begin{figure}[H]
            \centering
            \includegraphics[scale = 0.7]{scripts/tmp/pseudoresistors_IV.pdf}
                \end{figure}
        \column{.48\textwidth}
        \begin{figure}[H]
            \centering
            \includegraphics [scale = 0.7]{scripts/tmp/pseudoresistors_R.pdf}
        \end{figure}
    \end{columns}


\end{frame}

\begin{frame}{Architektura wzmacniacza neuronowego wykorzystującego sprzężenie zmiennoprądowe w różnych implementacjach pseudo-rezystorów}
    \begin{columns}[t]
        \column{.4\textwidth}
        \begin{alertblock}{Zmienne napięcia na bramce -- $variable-V_{gs}$}
            \begin{figure}[H]
                \centering
                \includegraphics[scale = 0.75]{ch3/fig1-ac_standard.pdf}
            \end{figure}
        \end{alertblock}



        \column{.4\textwidth}
        \begin{exampleblock}{Słałe napięcia na bramce -- $fixed-V_{gs}$}
            \begin{figure}[H]
                \centering
                \includegraphics[scale = 0.75]{ch3/fig1-ac_work_simplicite}
            \end{figure}
        \end{exampleblock}

    \end{columns}
\end{frame}

\begin{frame}{Analiza Transient sprzężenia AC}
    \begin{block}{Ustawienia symulacji}
\begin{itemize}
    \item Czeęstotliwość graniczna dla sprzężenia AC $\SI{\sim 1}{\hertz}$
    \item $\mathrm{THD} = \frac{\sqrt{\sum_{n=2}^{+\infty} U_k^2}}{U_1}$
\end{itemize}
    \end{block}


    \begin{columns}
        \column{.48\textwidth}
        \begin{figure}[H]
            \centering
            \includegraphics[scale = 0.7]{scripts/tranTHDAmp/tranTHDAmp_ner.pdf}
        \end{figure}
        \column{.48\textwidth}
        \begin{figure}[H]
            \centering
            \includegraphics[scale = 0.7]{scripts/tranTHDAmp/tranTHDAmp_pr_sim.pdf}
        \end{figure}
    \end{columns}

\end{frame}



\begin{frame}{Projekt przedwzmacniacza z modelem pseudo-rezystora w technologii $\SI{180}{\nano\metre}$  XFAB}
    \begin{figure}[H]
        \centering
        \includegraphics[scale=1.0]{ch3/pseudo-lay.pdf} 

    \end{figure}


\end{frame}

\begin{frame}{Wpływ pojemnościowych prądów bramki pseudo-rezystorów na zniekształcenia  w technologii $\SI{180}{\nano\metre}$}
    \vspace{-5mm} %5mm vertical space

    \begin{columns}
        \column{.3\textwidth}
        \begin{figure}[H]

            \includegraphics[trim={0 0.25cm 0 0.25cm}, clip, scale = 0.6]{scripts/tmp/IdsA.pdf}

        \end{figure}
        \column{.3\textwidth}
        \begin{figure}[H]

            \includegraphics[trim={0 0.25cm 0 0.25cm}, clip, scale = 0.6]{scripts/tmp/IdsB.pdf}

        \end{figure}
        \column{.3\textwidth}
        \begin{figure}[H]

            \includegraphics[trim={0 0.25cm 0 0.25cm}, clip, scale = 0.6]{scripts/tmp/IgbA.pdf}

        \end{figure}

    \end{columns}
    \vspace{-10mm} %5mm vertical space
    \begin{columns}
        \column{.3\textwidth}
        \begin{figure}[H]

            \includegraphics[trim={0 0.25cm 0 0.25cm}, clip, scale = 0.6]{scripts/tmp/IgbB.pdf}

        \end{figure}
        \column{.3\textwidth}
        \begin{figure}[H]

            \includegraphics[trim={0 0.25cm 0 0.25cm}, clip, scale = 0.6]{scripts/tmp/vOut.pdf}

        \end{figure}
        \column{.3\textwidth}
        \begin{figure}[H]

            \includegraphics[trim={0 0.25cm 0 0.25cm}, clip, scale = 0.6]{scripts/tmp/Igb_diff.pdf}

        \end{figure}

    \end{columns}
        



\end{frame}


\begin{frame}{Skalowanie zniekształceń z powierzchnią bramki i grubością tlenku tranzystorów tworzących pseudo-rezystory}
    \begin{columns}
        \column{.48\textwidth}
        \begin{block}{Powierzchnia bramki -- technologia $\SI{180}{\nano\metre}$ }
            \begin{figure}[H]
                \centering
                \includegraphics[scale = 0.7]{scripts/analyseTran/analyseTranSize.pdf}
            \end{figure}
        \end{block}

        \column{.48\textwidth}
        \begin{block}{Zależnosć od technologii}
            \begin{figure}[H]
                \centering
                \includegraphics[scale = 0.7]{scripts/analyseTran/analyseTranTechnology.pdf}
            \end{figure}
        \end{block}


    \end{columns}

\end{frame}

\begin{frame}{Szumy}
    \vspace{-5mm} %5mm vertical space

    \begin{columns}
    \column{.3\textwidth}
    \begin{figure}[H]
        \centering
        \includegraphics[scale=0.5]{ch2/conceptAC_Harrison.pdf} 
    \end{figure}
        \column{.3\textwidth}
        \begin{figure}[H]
            \centering
            \includegraphics[scale = 0.5]{scripts/tmp/fig3_R1.pdf}
        \end{figure}
        \column{.3\textwidth}
        \begin{figure}[H]
            \centering
            \includegraphics[scale = 0.5]{scripts/tmp/fig3_R2.pdf}
        \end{figure}
    \end{columns}

    \vspace{-5mm} %5mm vertical space


    \begin{columns}
        \column{.3\textwidth}
        \begin{figure}[H]
            \centering
            \includegraphics[scale=0.5]{scripts/noiseOutResistors/fig1_R1.pdf}
        \end{figure}
        \column{.3\textwidth}
        \begin{figure}[H]
            \centering
            \includegraphics[scale=0.5]{scripts/noiseOutResistors/fig2.pdf}
        \end{figure}
    \end{columns}


\end{frame}


\begin{frame}{Wpływ pojemności wejściowych na szumy i zniekształcenia}
    \begin{figure}[H]
        \centering
        \includegraphics{scripts/tmp/thd_C_in.pdf} 
    \end{figure}

\end{frame}


\part{Operacyjny wzmacniacz transkonduktancyjny}
% \begin{frame}{Implementacja teleskopowej kaskody ze zintegrowanym sprzężeniem AC}
%     \begin{columns}

%         \column{.6\textwidth}
%         \begin{figure}[H]
%             \includegraphics[scale = 0.75]{ch4/chap4Scheme.pdf} 
%         \end{figure}

%         \column{.35\textwidth}


%         \begin{block}{Kluczowe wymagnia}
%         \begin{itemize}
%             \item optymalizacja szumowa
%             \item powierzchnia
%             \item pobór mocy
%         \end{itemize}
%             \end{block}



%     \end{columns}   
%  \end{frame}



% \begin{frame}{Analiza szumowa pary różnicowej}
%     \begin{figure}[H]
%         \centering
%         \includegraphics[scale = 0.75]{scripts/tmp/differentialPair.pdf}  
%     \end{figure}

% \end{frame}



\begin{frame}{Przedwzmacniacz z wejściowym obwodem sprzęgającym AC}
    \begin{columns}

    \column{.65\textwidth}
    \begin{figure}[H]
        \centering
        \includegraphics[scale=0.45]{ch4/channel.pdf} 
    \end{figure}   

    \column{.35\textwidth}

    \begin{block}{
    }
    {\renewcommand\normalsize{\small}%
    \normalsize
    \begin{itemize}
        \item Konfiguracja  teleskopowej kaskody  jako aktywny OTA
        \item Polaryzacja pary różnicowej w obszarze podprogowym pracy tranzystora
        \item  Napięcie zasilania $\SI{\pm 1.8}{\volt}$
    \end{itemize}
    \vspace{-1em}
    \begin{table}[H]
        \centering
        \begin{tabular}{lll} 
        \toprule
        \begin{tabular}[c]{@{}l@{}}Kluczowe \\tranzystory\end{tabular} & $W$ [$\SI{}{\micro\metre}$] & $L$ [$\SI{}{\micro\metre}$]  \\ 
        \toprule
        $M_{bias}$                                                       & 10                          & 10                           \\
        $M_1,\ M_2$                                                    & 300                         & 1                            \\
        $M_3,\ M_4$                                                    & 20                          & 2                            \\
        $M_5,\ M_6$                                                    & 5                           & 5                            \\
        $M_7,\ M_8$                                                    & 4                           & 48                           \\
        \bottomrule
        \end{tabular}
    \end{table}
    }
    \end{block}
    \end{columns}   
  
\end{frame}


% \begin{frame}{Blok korekcji}
% \begin{columns}

%     \column{.35\textwidth}
%     \begin{block}{
%         Projekt kanału}
%         \begin{figure}[H]
%             \centering
%             \includegraphics[scale = 0.4]{ch4/chap4Scheme.pdf}
%         \end{figure} 
%         \end{block}

%         \begin{block}{
%             Wyzwania do rozwiązania}
%             \begin{figure}[H]
%                 \centering
%                 \includegraphics[scale = 0.6]{ch4/vgs_corr_sch.pdf} 
%             \end{figure}   
%             \end{block}



%     \column{.6\textwidth}
%     \begin{columns}
%     \column{.45\textwidth}

%     \begin{figure}[H]
%         \centering
%         \includegraphics[scale = 0.45]{scripts/tmp/analyseVgsTHD_1.pdf}
%     \end{figure} 
%     \column{.45\textwidth}
%     \begin{figure}[H]
%         \centering
%         \includegraphics[scale =0.45]{scripts/tmp/analyseVgsTHD_2.pdf}
%     \end{figure} 
%     \end{columns}   

%     \begin{columns}
%     \column{.45\textwidth}

%     \begin{figure}[H]
%         \centering
%         \includegraphics[scale = 0.4]{ch4/vgs_corr0.pdf}
%     \end{figure} 
%     \column{.45\textwidth}
%     \begin{figure}[H]
%         \centering
%         \includegraphics[scale = 0.4]{ch4/vgs_corr1.pdf}
%     \end{figure} 
%     \end{columns}   
% \end{columns}  
% \end{frame}


\begin{frame}{}
    \begin{columns}

    \column{.45\textwidth}
    \begin{block}{}
        {\renewcommand\normalsize{\small}%
        \normalsize
        \begin{itemize}
            \item 8 wersji przedwzmacniacza i 14 kanałów
            \item 4 wersje tranzystorów PMOS tworzących pseudo-rezystory  --  $W/L$: $2/40,\ 1/40,\ 2/20,\ 1/20\ \SI{}{\micro\metre / \micro\metre}$
            \item 2 konfiguracje pojemności -- $C_{in}/C_f = 4/200,\ 8/400\ \SI{}{\pico\farad}/\SI{}{\femto\farad}$
        \end{itemize}
        }
    \end{block}

\vspace{-1em}
    \begin{figure}[H]
        \centering
        \includegraphics[trim={0 12cm 0 0},clip, scale = 0.5]{ch4/layoutASIC.pdf} 
    \end{figure}   
    \column{.5\textwidth}

    \begin{block}{
Symulacje Post-Layout
    }

    \begin{figure}[H]
        \centering
        \includegraphics[scale = 0.45]{scripts/tranSchematicLayout/tranSchematicLayout.pdf}  
    \end{figure}
    \vspace{-5mm} %5mm vertical space
    \begin{figure}[H]
        \centering
        \includegraphics[scale = 0.45]{scripts/noiseContribution/noiseContributionOut.pdf}  
    \end{figure}
    \end{block}
    \end{columns}   
  
\end{frame}






\part{Weryfikacja elektroniczna i neurofizjologiczna układu scalonego HiFiNeuroPre}
\begin{frame}{System testowy}

    \begin{figure}[H]
        \centering
        \begin{subfigure}[b]{0.65\textwidth}
            \centering
            \includegraphics[width=\textwidth]{ch5/IMG_3725.jpg}
        \end{subfigure}
        \hfill
        \begin{subfigure}[b]{0.3\textwidth}
            \centering
            \includegraphics[width=\textwidth]{ch5/chip.jpg} 
            \includegraphics[width=\textwidth]{ch5/asic_photo.jpg}
        \end{subfigure}     

   \end{figure}
\end{frame}


\begin{frame}{Wzmocnienie, częstotliwość graniczna}

    \begin{figure}[H]
        \centering
        \begin{subfigure}[b]{0.485\textwidth}
            \centering
            \includegraphics{scripts/tmp/bodePlotFc_2.pdf}  

        \end{subfigure}
        % \hfill
        \begin{subfigure}[b]{0.485\textwidth}
            \centering
            \includegraphics{scripts/tmp/bodePlotFc_ictrl.pdf}

        \end{subfigure}     

    \end{figure}


\end{frame}

\begin{frame}{Jednorodność kanałów}
    \begin{figure}[H]
        \centering
        \begin{subfigure}[b]{0.485\textwidth}
            \centering
            \includegraphics{scripts/tmp/bodePlotFc_0.pdf}

        \end{subfigure}
        % \hfill
        \begin{subfigure}[b]{0.485\textwidth}
            \centering
            \includegraphics{scripts/tmp/bodePlotFc_1.pdf}

        \end{subfigure}     
    \end{figure}
\end{frame}

\begin{frame}{Pomiary zniekształceń harmonicznych -- wpływ korekty}

    \begin{columns}

        \column{.45\textwidth}
        \begin{block}{Brak globalnej korekty}
            \begin{figure}[H]
                \centering
                \includegraphics[scale = 0.85]{scripts/tmp/thdFreqCorr0_0.pdf} 
            \end{figure}   
        \end{block}

        \column{.45\textwidth}

        \begin{block}{Korekta globalna}
            \begin{figure}[H]
                \centering
                \includegraphics[scale = 0.85]{scripts/tmp/thdFreqCorr100_0.pdf}
            \end{figure}   
        \end{block}
    \end{columns}

\end{frame}

\begin{frame}{Konfiguracje przedwzmacniacza}


    \begin{columns}

        \column{.45\textwidth}
        \begin{block}{Symetryczna konfiguracja}


            \begin{figure}[H]
                \centering
                \includegraphics[trim={0 0.25cm 0 0.25cm}, clip, scale = 0.8]{scripts/embc2021THD_size/embc2021THD_size_0_100.pdf}
            \end{figure}   
        \end{block}

        \column{.45\textwidth}

        \begin{block}{Drugi wariant przedwzmacniacza z większymi pojemnościami wejściowymi}


            \begin{figure}[H]
                \centering
                \includegraphics[trim={0 0.25cm 0 0.25cm}, clip, scale = 0.8]{scripts/embc2021THD_size/embc2021THD_size_1_100.pdf}
            \end{figure}   
        \end{block}
    \end{columns}


    % \begin{figure}[H]
    %     \centering
    %     \begin{subfigure}{0.485\textwidth}
    %         \centering
    %         \includegraphics[trim={0 0.25cm 0 0.25cm}, clip, scale = 0.6]{scripts/embc2021THD_size/embc2021THD_size_0_0.pdf}
    %     \end{subfigure}
    %     % \hfill
    %     \begin{subfigure}{0.485\textwidth}
    %         \centering
    %         \includegraphics[trim={0 0.25cm 0 0.25cm}, clip, scale = 0.6]{scripts/embc2021THD_size/embc2021THD_size_1_0.pdf}
    %     \end{subfigure} 
    %     % \vfill
    %     \begin{subfigure}{0.485\textwidth}
    %         \centering
    %         \includegraphics[trim={0 0.25cm 0 0.25cm}, clip, scale = 0.6]{scripts/embc2021THD_size/embc2021THD_size_0_100.pdf}
    %     \end{subfigure}
    %     % \hfill
    %     \begin{subfigure}{0.485\textwidth}
    %         \centering
    %         \includegraphics[trim={0 0.25cm 0 0.25cm}, clip, scale = 0.6]{scripts/embc2021THD_size/embc2021THD_size_1_100.pdf}
    %     \end{subfigure}   
    % \end{figure}
\end{frame}

\begin{frame}{Pomiary szumów}
    % \begin{figure}[H]
    %     \centering 
    %     \includegraphics[scale=0.4]{scripts/tmp/measurementNoiseDataset.pdf}  
    % \end{figure}

    \begin{figure}[H]
        \centering
        \begin{subfigure}[b]{0.485\textwidth}
            \centering
            \includegraphics{scripts/tmp/noiseGND_in.pdf}
        \end{subfigure}
        % \hfill
        \begin{subfigure}[b]{0.485\textwidth}
            \centering
            \includegraphics{scripts/tmp/noiseElektrodaNaCl.pdf}
        \end{subfigure}     
    \end{figure}
\end{frame}



\begin{frame}{System pomiarowy do akwizycji sygnałów neurobiologicznych}
    \begin{figure}[H]
        \centering 
        \includegraphics[scale=0.25]{ch6/setupIBDtot.png}  
    \end{figure}
\end{frame}

\begin{frame}{}
    \begin{figure}[H]
       % \begin{subfigure}{0.3\textwidth}
   
       %  \end{subfigure}
       %      \hfill
   
       \begin{subfigure}{0.25\textwidth}
           \includegraphics[scale = 0.75]{ch6/meaLFPnexus.pdf}
            % \vfill
        \end{subfigure}
       % \hfill
           \hspace{-5em}
        \begin{subfigure}{0.7\textwidth}
           \includegraphics[scale = 0.75]{scripts/tmp/signal_MEA_LFP_wide.pdf}
        \end{subfigure}
   \end{figure}
\end{frame}

\begin{frame}{}
    \begin{figure}[H]
        \centering
        \begin{subfigure}[b]{0.485\textwidth}
            \centering
            \includegraphics[scale=0.8]{scripts/tmp/signal_MEA_AP_1.pdf}
            \caption{}
        \end{subfigure}
        % \hfill
        \begin{subfigure}[b]{0.485\textwidth}
            \centering
            \includegraphics[scale=0.8]{scripts/tmp/signal_MEA_AP_2.pdf}
            \caption{}
        \end{subfigure}     
    \end{figure}
\end{frame}


\part{Podsumowanie}
\begin{frame}{Podsumowanie testów elektronicznych}

\begin{longtblr}[
    caption = {Parametry przedwzmacniacza na podstawie pomiarów weryfikacyjnych}
    % label = {table:paramResult},
  ]{
    hline{1-2,11} = {-}{0.08em},
  }
  \textbf{Parametr}                                                                 & \textbf{Wartość}                    \\
  Napięcia zasilania                                                                & $\SI{\pm 1.8}{\volt}$               \\
  Całkowity prąd                                                                    & $\SI{2}{\micro\ampere}$             \\
  Pobór mocy dla pojedynczego kanału                                                & $\SI{7.2}{\micro\watt}$             \\
  Wzmocnienie z zamkniętą pętlą sprzężenia                                          & $\SI{25.9}{\deci\bel}$              \\
  Zakres dolnej częstotliwości granicznej                                           & $\SIrange{0.1}{20}{\hertz}$         \\
  Ekwiwalentny szum wejściowy w zakresie LFP                                        & $\SI{7.5}{\micro\volt_{rms}}$       \\
  Ekwiwalentny szum wejściowy w zakresie AP                                         & $\SI{6.7}{\micro\volt_{rms}}$       \\
  Zniekształcenia harmonioczne THD – $\SI{10}{\milli\volt_{pp}}\ \SI{1.68}{\hertz}$ & $\SI{0.94}{\percent}$               \\
  Pole powierzchni pojedynczego przedwzmacniacza                                    & $\SI{0.0071}{\milli\metre\squared}$ 
  \end{longtblr}
\end{frame}

\begin{frame}{Wnioski}
\begin{itemize}
    \item 	
\end{itemize}
\end{frame}
% % \ifdefined\textleftmargin
% 	%%%%%%%%%%%%%%%%
% 	\begin{frame}[fragile]{\iflanguage{polish}{Informacje}{Information}}
% 		\begin{center}
% 			\iflanguage{polish}{
% 				$\Longleftarrow$ Aktualna wartość \hfill Aktualna wartość $\Longrightarrow$\\
% 				$\Longleftarrow$ rozmiaru lewego marginesu \hfill rozmiaru prawego marginesu $\Longrightarrow$\\
% 				$\Longleftarrow$ to \the\textleftmargin \hfill to \the\textrightmargin $\Longrightarrow$
% 			}{
% 				$\Longleftarrow$ The current value of \hfill The current value of $\Longrightarrow$\\
% 				$\Longleftarrow$ the  left  margin size \hfill the right margin size $\Longrightarrow$\\
% 				$\Longleftarrow$ is \the\textleftmargin \hfill is \the\textrightmargin $\Longrightarrow$
% 			}
% 		\end{center}
% 		\iflanguage{polish}{
% 			Możesz je zmieniać za pomocą parametru 'margins' \pauza
% 		}{
% 			You can change them with the 'margins' parameter ---
% 		}
% 		\verb+\usetheme[margins=...]{AGH}+
% 	\end{frame}
% \fi
%%%%%%%%%%%%%%%%

%%%%%%%%%%%%%%%%
\begin{frame}{\iflanguage{polish}{Plan prezentacji}{Outline}}
	\tableofcontents[pausesections]
\end{frame}
%%%%%%%%%%%%%%%%
\section{\iflanguage{polish}{Elementy podstawowe}{Basic elements}}
%%%%%%%%%%%%%%%%
\begin{frame}{\iflanguage{polish}{Wyszczególnienie}{Itemize}}
	\begin{columns}
		\column{0.5\textwidth}
		\begin{itemize}
			\item \iflanguage{polish}{Element 1}{Item 1}
			\item \iflanguage{polish}{Element 2}{Item 2}
			\item \iflanguage{polish}{Element 3}{Item 3}
		\end{itemize}
		\column{0.5\textwidth}
		\pause
		\structure{\iflanguage{polish}{Odkrywanie po kolei}{Uncovering one by one}}
		\begin{itemize}[<+->]
			\item \iflanguage{polish}{Element 1}{Item 1}
			\item \iflanguage{polish}{Element 2}{Item 2}
			\item \iflanguage{polish}{Element 3}{Item 3}
		\end{itemize}
		\onslide
	\end{columns}
\end{frame}
%%%%%%%%%%%%%%%%
\begin{frame}{\iflanguage{polish}{Wyliczenie}{Enumerate}}
	\begin{columns}
		\column{0.5\textwidth}
		\begin{enumerate}
			\item \iflanguage{polish}{Element 1}{Item 1}
			\item \iflanguage{polish}{Element 2}{Item 2}
			\item \iflanguage{polish}{Element 3}{Item 3}
		\end{enumerate}
		\column{0.5\textwidth}
		\pause
		\structure{\iflanguage{polish}{Odkrywanie elementów po kolei z jednoczesnym wyróżnianiem}{Uncovering elements in turn with simultaneous highlighting}}
		\begin{enumerate}[<+-|alert@+>]
			\item \iflanguage{polish}{Element 1}{Item 1}
			\item \iflanguage{polish}{Element 2}{Item 2}
			\item \iflanguage{polish}{Element 3}{Item 3}
		\end{enumerate}
		\onslide
	\end{columns}
\end{frame}
%%%%%%%%%%%%%%%%
\section{\iflanguage{polish}{Matematyka}{Mathematics}}
%%%%%%%%%%%%%%%%
\begin{frame}{\iflanguage{polish}{Podstawowe bloki}{Basic blocks}}
	% Examples from "The beamer class User Guide"
	\iflanguage{polish}{
		\begin{block}{Definicja}
			\alert{Zbiór} składa się z elementów.
		\end{block}
		\begin{exampleblock}{Przykład}
			Zbiór $\{1,2,3,5\}$ zawiera cztery elementy.
		\end{exampleblock}
		\begin{alertblock}{Błędne Twierdzenie}
			$1=2$.
		\end{alertblock}
	}{
		\begin{block}{Definition}
			A \alert{set} consists of elements.
		\end{block}
		\begin{exampleblock}{Example}
			The set $\{1,2,3,5\}$ has four elements.
		\end{exampleblock}
		\begin{alertblock}{Wrong Theorem}
			$1=2$.
		\end{alertblock}
	}
\end{frame}
%%%%%%%%%%%%%%%%
\begin{frame}{\iflanguage{polish}{Otoczenia matematyczne}{Math environments}}
	\begin{columns}
		\column{0.45\textwidth} %The first column
		\structure{\iflanguage{polish}{Twierdzenia}{Theorems}}
		\begin{theorem}[\iflanguage{polish}{Pitagorasa}{Pythagorean}]
			$a^{2}+  b^{2}=  c^{2}$
		\end{theorem}
		\column{0.45\textwidth} %The second column
		\structure{\iflanguage{polish}{Dowody}{Proofs}}
		\begin{proof}
			\ldots
		\end{proof}
	\end{columns}
	\vfill
	\ldots
	\vfill
	\begin{definition}
		\ldots
	\end{definition}
\end{frame}
%%%%%%%%%%%%%%%%
\begin{frame}{\iflanguage{polish}{Dynamiczny wzór matematyczny}{Dynamic mathematical formula}}
	\[
		\binom{n}{k} = \pause \frac{n!}{k!(n-k)!}
	\]
\end{frame}
%%%%%%%%%%%%%%%%
% \section{\iflanguage{polish}{Informatyka}{Computer Science}}
% %%%%%%%%%%%%%%%%
% \subsection*{\iflanguage{polish}{Wstawianie kodów źródłowych}{Inserting source codes}}
% %%%%%%%%%%%%%%%%
% \begin{frame}[fragile]{\iflanguage{polish}{Użycie otoczenia 'listings'}{Using the 'listings' environment}}
% 	\begin{lstlisting}[language=C++]
% /* The first  program in C++ */  %*\pause*)
% #include <iostream>  %*\pause*)
% using namespace std; %*\pause*)
% void main() 
% {       %*\pause*)
%   cout %*\pause*) << "Hello World!"%*\pause*) << endl; %*\onslide<4->*)
% } %*\onslide*)
% \end{lstlisting}
% \end{frame}
%%%%%%%%%%%%%%%%
% \begin{frame}[fragile]{\iflanguage{polish}{Użycie otoczenia 'minted'}{Using the 'minted' environment}}
% 	\begin{minted}[beameroverlays,escapeinside=||]{C++}
% /* The first  program in C++ */  |\pause|
% #include <iostream>  |\pause|
% using namespace std; |\pause|
% void main() 
% {       |\pause|
%   cout |\pause| << "Hello World!"|\pause| << endl; |\onslide<4->|
% } |\onslide|
% 	\end{minted}
% \end{frame}
% %%%%%%%%%%%%%%%%%%%%%%%
% \appendix
% %%%%%%%%%%%%%%%%%%%%%%%
% \begin{frame}[allowframebreaks]{\iflanguage{polish}{Bibliografia}{Bibliography}}
% 	\begin{thebibliography}{9}
% 		\setbeamertemplate{bibliography item}[online]
% 		\bibitem{wikibook}{Wikibooks \newblock \LaTeX/Source Code Listings \newblock \url{https://en.wikibooks.org/wiki/LaTeX/Source_Code_Listings}}
% 		\bibitem{beamer}{Till Tantau, Joseph Wright, Vedran Miletić \newblock The beamer class \newblock \url{http://mirror.ctan.org/macros/latex/contrib/beamer/doc/beameruserguide.pdf}}
% 		\setbeamertemplate{bibliography item}[book]
% 		\bibitem{lamport}{Leslie Lamport \newblock LATEX: a document preparation system : user's guide and reference manual \newblock Addison-Wesley Pub. Co., 1994 }
% 		\setbeamertemplate{bibliography item}[article]
% 		\iflanguage{polish}{
% 			\bibitem{article1}{Autor \newblock Tytuł artykułu \newblock Edytor, rok \newblock Uwagi}
% 			\setbeamertemplate{bibliography item}[triangle]
% 			\bibitem{article2}{Autor \newblock Tytuł artykułu \newblock Edytor, rok \newblock Uwagi}
% 			\setbeamertemplate{bibliography item}[text]
% 			\bibitem{article3}{Autor \newblock Tytuł artykułu \newblock Edytor, rok \newblock Uwagi}
% 			\bibitem[Polak98]{article4}{Autor \newblock Tytuł artykułu \newblock Edytor, rok \newblock Uwagi}
% 		}{
% 			\bibitem{article1}{Author \newblock Title of the article\newblock Editor, year \newblock Notes}
% 			\setbeamertemplate{bibliography item}[triangle]
% 			\bibitem{article2}{Author \newblock Title of the article\newblock Editor, year \newblock Notes}
% 			\setbeamertemplate{bibliography item}[text]
% 			\bibitem{article3}{Author \newblock Title of the article\newblock Editor, year \newblock Notes}
% 			\bibitem[Polak98]{article4}{Author \newblock Title of the article\newblock Editor, year \newblock Notes}
% 		}
% 	\end{thebibliography}
% \end{frame}
\end{document}
