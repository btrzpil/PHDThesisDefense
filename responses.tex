%%% Pfitzner
\begin{frame}[t]
    
    \begin{block}{\af}
        \cit{
            Autorka nie sformułowała tezy rozprawy, co wynika ze specyfiki pracy, natomiast celem praktycznym było opracowanie takiego rozwiązania konstrukcyjnego przedwzmacniacza, aby umożliwić skuteczną realizację zintegrowanego narzędzia do badań mózgu przez zespół Katedry Oddziaływań i Detekcji Cząstek AGH.
        }
    \end{block}

\end{frame}

\begin{frame}[t]
    \begin{block}{\af}
        \cit{
            W tekście pracy jest relatywnie dużo literówek, występują też drobne błędy gramatyczne, co świadczy prawdopodobnie o nadmiernie pospiesznym finalizowaniu rozprawy. 
            Innych niedociągnięć jest niewiele, np.: na rysunkach 3.13 i 3.14 na osi odciętych skala częstotliwości obejmuje zakres od 0.1 Hz do 100 Hz, podczas gdy w podpisie podano pasmo 1 Hz -- 10 kHz. 
            Ponadto tabela 4.3 jest wadliwie zbudowana i bez dodatkowego opisu nieczytelna.
        }
    \end{block}

\end{frame}

\begin{frame}[t]
    \begin{block}{\af}
        \cit{
            Zwykle jednak wymagania projektowe formułowane są explicite w formie granicznych wartości istotnych parametrów, lecz takich na dla pracy nie określono. 
            W tej sytuacji pierwszorzędnego znaczenia nabiera porównanie wartości osiągniętych parametrów zaprojektowanego wzmacniacza z danymi dostępnymi z literatury. 
            Brak zestawienia porównawczego w formie tabeli budzi pewien niedosyt. Wprawdzie standardowo w odniesieniu do zniekształceń harmonicznych wartość THD podawana jest zwykle dla częstotliwości 1 kHz, a nie w szerszym zakresie, to jednak takie zestawienie byłoby zasadne uwzględniając zarówno ogólne stwierdzenia jak i wartości innych parametrów.
        }
    \end{block}
\end{frame}

%%% Komorowski

\begin{frame}[t]
    \begin{block}{\dk}
        \cit{
            W przypadku przedstawionej pracy Autorka precyzyjnie przedstawiła główny cel pracy oraz wyszczególniła kolejny etapy pracy prowadzące do realizacji celu głównego.
            Niestety w pracy nie zauważyłem wyszczególnionych tez pracy. 
            \high{
                Jakie są zatem cele pracy? Czy tezy pracy zostały udowodnione?
            }
        }
    \end{block}
\end{frame}

\begin{frame}[t]
    \begin{block}{\dk}
        \cit{
            Omówiono również inne koncepcje eliminacji składowej stałej: technikę wzmacniaczy z modulacją [...] oraz zastosowanie wzmacniacza o małym lub jednostkowym wzmocnieniu i przetwornika analogowo-cyfrowego o dużej rozdzielczości.
            Z uwagi na rosnącą popularność tego ostatniego rozwiązania w przypadku akwizycji sygnałów biologicznych i biomedycznych uważam, że ta część rozdziału powinna być bardziej szczegółowa. 
        }
    \end{block}
\end{frame}

\begin{frame}[t]
    \begin{block}{\dk}
        \cit{
            Struktura rozprawy jest logiczna, jednak w pracy występuje pewna (moim zdaniem stanowczo zbyt liczna) liczba uchybień edycyjnych (tekstowych, językowych), np. forma gramatyczna, powtórzenia, brak przedimków lub słów itp.
        }
    \end{block}
\end{frame}

\begin{frame}[t]
    \begin{block}{\dk}
        \cit{
            W pracy występuje dość obszerna część medyczno-biologiczna służąca w zasadzie do uzasadnienia podziału sygnałów na dwie grupy LFP i AP o odpowiednich charakterystykach (pasmo i zakres amplitud).
            Materiał ciekawy i ważny z punktu widzenia pracy, ale być może mógłby być krótszy.
        }
    \end{block}
\end{frame}

\begin{frame}[t]
    \begin{block}{\dk}
        \cit{
            Niektóre używane w pracy określenia moim zdaniem są nietypowe, głównie dotyczy to określenia "\high{układ odczytu}" zamiast wzmacniacz.
            [...]
            Podobnie użycie słowa "\high{zaadresowany}".
            Wyrażenie "\high{mniejsze multipleksery}" (str. 34) jest mało precyzyjne, a właściwie w użytym kontekście nieprawidłowe.
            Również definicja wyrażenia "\high{front-end}" (str. 35) budzi moje wątpliwości.
            Użycie wyrażenia "\high{może zostać drastycznie zmniejszona}" wydaje mi się również niezbyt fortunne.
            Autorka rozprawy dosyć często używa wyrażenia \high{offset wyjściowy}, moim zdaniem jednak poprawnie powinno się użytwać wyrażenia offset napięcia wyjściowego.
        }
    \end{block}
    \begin{columns}

        \column{.33\textwidth}
        \begin{figure}[H]
            \includegraphics[scale = 0.2]{ch2/afe.png}
        \end{figure}
    
        \column{.6\textwidth}
        Schematy rodzajów architektur systemów wielokanałowych do rejestracji biosygnałów elektrycznych. 
        W pierwszym rozwiązaniu jedno ADC używane jest do wszystkich
        kanałów, w kolejnym rozwiązaniu każdy kanał rejestrujący posiada własne ADC, ostatnie rozwiązanie stanowi połączenie obu koncepcji zakładające jedno ADC dla jednej kolumny LNA
    \end{columns}

    
\end{frame}

\begin{frame}[t]
    \begin{block}{\dk}
        \cit{
            \high{
                Impedancja mikroelektrody jest ważnym parametrem dla rejestracji zewnątrzkomórkowej, ponieważ określa szumy elektrody oraz tłumienie sygnału.} -- w jaki sposób impedanca określa te parametry?
        }
    \end{block}
    \begin{columns}

        \column{.43\textwidth}
        \begin{figure}[H]
            \includegraphics[scale = 0.6]{ch2/ImpedancjaWykres.pdf}
        \end{figure}
        \begin{figure}[H]
            \includegraphics[scale = 0.2]{ch2/chemNeuroInterface.png}
        \end{figure}
        
    
        \column{.55\textwidth}
        Zależność impedancji elektrody $Z$ od częstości sygnału

    \end{columns}
\end{frame}

\begin{frame}[t]
    \begin{block}{\dk}
        \cit{
            \high{Wzmacniacz LNA występuje w każdym kanale, skąd następnie sygnał jest przesyłany poprzez MUX z podziałem czasu.
            Wadą tego rozwiązania jest to, że gdy liczba kanałów wzrasta, częstotliwość próbkowania ADC również wzrasta, co powoduje większy pobór mocy.} - Częstotliwość próbkowania powinna być zależna od właściwości próbkowanego (rejestrowanego) sygnału, a nie zależeć od architektury systemu.
        }
    \end{block}
W przypadku systemów z multipleksem o podziale czasowym (TDM), częstotliwość próbkowania musi być zazwyczaj wyższa niż minimalna częstotliwość Nyquista wynikająca z charakterystyki pojedynczego kanału. Jest to spowodowane faktem, że każdy kanał jest próbkowany w sekwencji, co wprowadza dodatkowe wymagania dotyczące częstotliwości próbkowania.

Oto dlaczego częstotliwość próbkowania musi być wyższa niż minimalna częstotliwość Nyquista:

Częstotliwość Nyquista dla pojedynczego kanału: Minimalna częstotliwość próbkowania wynika z zasady Nyquista i jest określona przez maksymalną częstotliwość sygnału w pojedynczym kanale. Jeśli ta częstotliwość jest na poziomie, na przykład, 10 kHz, to minimalna częstotliwość próbkowania wynosiłaby 20 kHz zgodnie z zasadą Nyquista.

Multipleksowanie kanałów: Jednak w systemie TDM sygnały z różnych kanałów są próbkowane w określonym czasie w sekwencji. Oznacza to, że system musi być w stanie przetworzyć dane z każdego kanału w odpowiednim czasie. Dlatego potrzebna jest wyższa częstotliwość próbkowania, aby zebrać dane z każdego kanału zgodnie z cyklem TDM.

Przepustowość systemu: Wyższa częstotliwość próbkowania oznacza większą ilość danych do przetworzenia i przesłania. Przepustowość systemu, zarówno sprzętowa, jak i komunikacyjna, musi być wystarczająca, aby obsłużyć te dane.

Zasoby sprzętowe i przetwarzanie danych: Wybór częstotliwości próbkowania musi być dostosowany do dostępnych zasobów sprzętowych i zdolności przetwarzania danych systemu.

Dlatego częstotliwość próbkowania w systemie TDM musi być wyższa, aby uwzględnić dodatkowe wymagania wynikające z próbkowania wielu kanałów w sekwencji. Wybór odpowiedniej częstotliwości próbkowania w takim systemie jest istotny, aby zapewnić dokładność pomiarów i niezakłócone przetwarzanie danych z każdego kanału.
\end{frame}

\begin{frame}[t]
    \begin{block}{\dk}
        \cit{
            Niektóre fragmenty tekstu są dla mnie niezrozumiałe lub budzą pewne wątpliwości
            [...]
            \high{
                Z punktu widzenia minimalizacji poboru mocy najkorzystniejsza jest polaryzacja tranzystorów w zakresie słabej inwersji, poniważ w tym zakresie transkonduktancji do prądu polaryzacji tranzystora jest największy [48].
            }

        }
    \end{block}
    \begin{columns}

        \column{.43\textwidth}
        \begin{figure}[H]
            \includegraphics[scale=0.25]{ch3/gmid_id.png} 
        \end{figure}
    
        \column{.55\textwidth}
        Stosunek transkonduktancji do prądu drenu w funkcji znormalizowanego prąd drenu dla trzech technologii 
    \end{columns}

\end{frame}

\begin{frame}[t]
    \begin{block}{\dk}
        \cit{
            Niektóre fragmenty tekstu są dla mnie niezrozumiałe lub budzą pewne wątpliwości
            [...]
            \high{ale przy stałym stosunku $C_{in}/C_{f} = 20 V/V$}
        }
    \end{block}

\end{frame}

\begin{frame}[t]
    \begin{block}{\dk}
        \cit{
            Niektóre fragmenty tekstu są dla mnie niezrozumiałe lub budzą pewne wątpliwości
            [...]
            \high{
                Jak wspomniano wcześniej, w docelowym rozwiązaniu przewiduje się zastosowanie drugiego stopnia wzmacniającego. 
                Przy założeniu, że kolejny stopień będzie miał wysoką impedancję wyjściową, może on być sterowany bezpośrednio z kaskody o wysokiej impedancji wyjściowej. 
                Dla celów testowych potrzebujemy jednak stopnia wyjściowego o relatywnie niskiej impedancji wyjściowej, który skutkowałby zwiększeniem poboru mocy układu prototypowego.
            }
        }
    \end{block}


    \begin{columns}

        \column{.43\textwidth}
        \begin{figure}[H]
            \includegraphics[scale = 0.4]{ch4/sf.pdf}
        \end{figure}

        \column{.55\textwidth}

        By móc utrzymać wysokie wzmocnienie różnicowe prądu stałego oferowane przez przedwzmacniacz niezależnie od dalszych elementów toru odczytowego powinien on zapewniać niską rezystancję wyjściową ze względu na możliwość obciążenia parametrów przedwzmacniacza.
        Aby układ przedwzmacniacza miał niską rezystancję wyjściową należało dodać stopień buforujący do wyjścia OTA.



    \end{columns} 

\end{frame}
    % Zaproponowany przedwzmacniacz jest pierwszym fragmentem toru odczytowego do rejestracji sygnałów neuronalnych. 
    % By móc utrzymać wysokie wzmocnienie różnicowe prądu stałego oferowane przez przedwzmacniacz niezależnie od dalszych elementów toru odczytowego powinien on zapewniać niską rezystancję wyjściową ze względu na możliwość obciążenia parametrów przedwzmacniacza.
    % Rezystancja wyjściowa dla OTA jest odwrotnością konduktancji gds, która jest w przybliżeniu proporcjonalna do prądu drenu par różnicowych. 
    % Aby układ przedwzmacniacza miał niską rezystancję wyjściową należało dodać stopień buforujący do wyjścia OTA.
    % Zdecydowano się wykorzystać do tego prosty wzmacniacz w konfiguracji wtórnika,który skonstruowany jest z tranzystora PMOS w konfiguracji wspólnego drenu (MSF) polaryzowanego za pomocą obciążenia aktywnego (tranzystoryMpolar). 
    % Wtórnik wyjściowy ze względu na obciążenie aktywne wymaga sygnału prądowego Ipolar co przekłada się na pobór mocy całego układu.



\begin{frame}[t]
    \begin{block}{\dk}
        \cit{
            Wzór 2.2 na str. 88 -- brak liczby 4 w mianowniku pod pierwiastkiem, nie wszystkie składowe wzoru są wyjaśnione i opisane.
        }
    \end{block}
\end{frame}

\begin{frame}[t]
    \begin{block}{\dk}
        \cit{
            W pracy przyjęto wzmocnienie dla pierwszego stopnia projektowanego wzmacniacza na poziomie $K = 20 V/V$.
            Czy w kontekście możliwości pojawienia się składowej stałej napięcia na wejściu wzmacniacza spowodowanego zjawiskami zachodzącymi na styku tkanka-elektroda wartość ta nie jest zbyt duża i czy nie będzie powodowała nasycenia stopnia wejściowego wzmacniacza?
        }
    \end{block}
    \begin{figure}[H]
        \centering
        \includegraphics[scale=0.5]{ch2/conceptAC_Harrison.pdf} 
    \end{figure}
\end{frame}

\begin{frame}[t]
    \begin{block}{\dk}
        \cit{
            W pracy skupiono się na analizie własności i projektowaniu rezystora półprzewodnikowego, nieco mniej zajmując się samym wzmacniaczem -- który jest najważniejszym elementem pracy.
            W szczególności dotyczy to parametrou CMRR wzmacniacza.
            Podobnie niewiele uwagi poświęconu napięciu offsetu wzmacniacza, chociaż jego obecność jest widoczna we wszystkich zarejestrowanych przebiegach.
            Tu przydatne byłoby jakieś oszacowanie.
            Konsekwencją takiego podejścia jest dość zwięzły opis samej struktury wzmacniacza i jego własności tu przydałaby się nieco bardziej obszerna analiza.
        }
    \end{block}
\end{frame}

\begin{frame}[t]
    \begin{block}{\dk}
        \cit{
            Wybór współczynnika THD do oceny parametrów wzmacniacza jest poprawny, ale w mojej opinii w pracy trochę za mało uwagi poświęcono innym, dość istotnym parametrom wzmacniacza mających wpływ na jakość rejestrowanych sygnałów np. takich jak liniowość fazy, odpowiedź na skok jednostkowy czy szybkość narastania (SR -- ang. slew rate).
        }
    \end{block}
\end{frame}

\begin{frame}[t]
    \begin{block}{\dk}
        \cit{
            Teza o dużym znaczeniu współczynnika THD dla niskoczęstotliwościowych składowych sygnałów nie jest poparta odpowiednimi przykładami uzasadniającymi to znaczenie.
            Jednak wymagałoby to dokładniejszej analizy własności rejestrowanych sygnałów neuronalnych, co jednak wykracza poza zakres pracy.
        }
    \end{block}
\end{frame}

\begin{frame}[t]
    \begin{block}{\dk}
        \cit{
            Zastanawiający jest brak w analizach widmowych zarejestrowanych sygnałów, składowych sieci i ich harmonicznych.
            Być może zostały użyte filtry typu \high{notch}, ale nie wspomniano o tym w pracy.
        }
    \end{block}

    \begin{figure}[H]
        \centering
        \begin{subfigure}[b]{0.485\textwidth}
            \centering
            \includegraphics[scale = 0.75]{scripts/tmp/noiseGND_in.pdf}
        \end{subfigure}
        % \hfill
        \begin{subfigure}[b]{0.485\textwidth}
            \centering
            \includegraphics[scale = 0.75]{scripts/tmp/noiseElektrodaNaCl.pdf}
        \end{subfigure}     
    \end{figure}
\end{frame}

%%% Buchner

\begin{frame}[t]
    \begin{block}{\tb}
        \cit{
            Nie jest zachowany klasyczny układ publikacji naukowej, natomiast zaproponowany układ jest logiczny i przejrzysty.
        }
    \end{block}
    Co to znaczy klasyczny układ publikacji naukowej?
    Nature
    IEEE
\end{frame}

\begin{frame}[t]
    \begin{block}{\tb}
        \cit{
            Protokół badania nie jest dokładnie opisany, ale zakłada badanie odpowiedzi wywołanej na mechaniczne drażnienie wibrys u szczura, z jednoczesną rejestracją aktywności LFP w obszarze od kory mózgowej (Cx) do wnętrza mózgu (Th).
        }
    \end{block}
\end{frame}

\begin{frame}[t]
    \begin{block}{\tb}
        \cit{
            Język rozprawy jest bardzo dobry, odnotowano jedynie nieliczne przypadki łączenia imiesłowu przysłówkowego ze stroną bierną.
        }
    \end{block}
\end{frame}

\begin{frame}[t]
    \begin{block}{\tb}
        \cit{
            natomiast rysunki stanowiące autocytaty z prac, których współautorką jest Autorką nie są oznaczone jako takie.
        }
    \end{block}
\end{frame}

\begin{frame}[t]
    \begin{block}{\tb}
        \cit{
            Użycie bibliografii jest nieco utrudnione przez brak sortowania alfabetycznego.
        }
    \end{block}
\end{frame}

\begin{frame}[t]
    \begin{block}{\tb}
        \cit{
            Cytowanie na ogół jest poprawne, jedynie cztery źródła to źródła internetow, dla których jedynym adresem publikacyjnym jest strona www. 
            Podane są one [...] bez daty dostępu, co jest błędem, jednak ich charakter nie wskazuje na ulotność ponieważ są to w większości strony firmowe, zawierające charakterystyki produktów.
        }
    \end{block}
\end{frame}

\begin{frame}[t]
    \begin{block}{\tb}
        \cit{
            Kod źródłowy skryptów wypracowanych w ramach pracy nie jest częścią rozprawy ani nie jest dostępny w publicznym repozytorium, choś publikacja kodu pomogłaby w ocenie kompetencji Autorki.
        }
    \end{block}
\end{frame}

\begin{frame}[t]
    \begin{block}{\tb}
        \cit{
            Metodykę procesu badawczego należy podzielić na dwa etapy [...]. 
            Pierwszy z tych etapów nie budzi zasadniczych wątpliwości. 
            Jedyny zidentyfikowany brak dotyczy wspomnianych na str. 68 wolnozmiennych oscylacji, które nie zostały uwzględnione w scenariuszach testowych, a jak wspomniano wcześniej w tekście, mają istotne znacznie dla działalności wzmancniacza.
        }
    \end{block}
\end{frame}

\begin{frame}[t]
    \begin{block}{\tb}
        \cit{
            Pewne wątpliwości budzi natomiast metodyka procesu weryfikacji w eksperymencie neurofizjologicznym.
            W przypadku pomiarów o charakterze unikatowym nie ma możliwości porównania wyniku z urządzeniami referencyjnymi.
            Wydaje się jednak, że ten przypadek tu nie zachodzi.
            Bardziej właściwe wydaje się porównanie omawianego urządzenia pomiarowego z urządzeniem referencyjnym w sposób, który wykaże prawidłowość realizowanych za jego pomocą pomiarów.
            Istnieją również wspierające ten proces metody statystyczne, takie jak metoda Blanda-Altmana.
        }
    \end{block}
\end{frame}

\begin{frame}[t]
    \begin{block}{\tb}
        \cit{
            W przypadku niniejszej pracy zastosowano jakościową metodę weryfikacji, którą jest zgodność otrzymanych z użyciem urządzenia wyników z oczekiwaniami eksperymentatora.
            Oczekiwania te zbudowane są na podstawie dostępnej wiedzy, a ta zawiera oczywiście wyniki pomiarów, które można uznać za referencyjne.
            Jednak nie ulega wątpliwości, że tego typu analiza jest dużo słabsza z perspektywy matematycznej niż analiza porównawcza dwóch urządzeń korzystających z tego samego źródła sygnału.
        }
    \end{block}
\end{frame}

\begin{frame}[t]
    \begin{block}{\tb}
        \cit{
            W tekście cytowanych jest kilka rodzin modeli symulacyjnych tranzystora MOS, należy się domyślać, że w dalszym ciągu użyty został model EKV.
        }
    \end{block}
\end{frame}

\begin{frame}[t]
    \begin{block}{\tb}
        \cit{
            Uwaga co do modelowania, wynikająca z rozważań na str. 60 jest taka, że konsekwencją zależności $C_{gb}$ od $V_{gb}$ jest obecność wyrazu kwadratowego w zależności $I_{gb}(U_{gb})$ -- warto rozważyć cząstkową publikację tego wyniku.
        }
    \end{block}
\end{frame}

\begin{frame}[t]
    \begin{block}{\tb}
        \cit{
            wykres przedstawiony na rys. 2.4 przedstawia filtr górnoprzepustowy a nie dolnoprzepustowy (wysoka impedancja występuje w paśmie niskich częstotliwości, a impedancja w paśmie wysokich częstości jest niska).
        }
    \end{block}
\end{frame}

\begin{frame}[t]
    \begin{block}{\tb}
        \cit{
            Warto również odnotować niekonsekwentny charakter opisu procesów fizykochemicznych zachodzących po stronie tkanki, które czasem opisywane są jako procesy jonowe, czasem jako procesy elektrochemiczne, a czasem odnoszone są do pojęcia warstwy podwójnej i efektów pojemnościowych.
            Co do zasady użyte modele matematyczne nie budzą wątpliwości, choć model matematyczny tkanki jest zdawkowy [...]
        }
    \end{block}
\end{frame}

\begin{frame}[t]
    \begin{block}{\tb}
        \cit{
            w analizie stanu sztuki Autorka pomija dorobek jej macierzystego zespołu, chociaż go cytuje [...].
            W związku z tym recenzent skazany jest na domysły.
            Należy przyjąć, że istotną nowością omawianej pracy jest użycie pseudorezystorów, ponieważ ta technika nie pojawia się w tytułach wymienionych [...] powyżej pozycji dorobku.
        }
    \end{block}
\end{frame}

\begin{frame}[t]
    \begin{block}{\tb}
        \cit{
            Autorka opisuje proces inżynierski z perspektywy ex post. 
            W związku z tym zdarza się, że relacjonując wykonane badanie czy analizę nie umieszcza na końcu rozdziału wniosków z tego badania.
            Pojawiają się one niejako mimochodem jako uzasadnienie decyzji projektowej, której podjęcie jest ralaconowane w rozdziale następnym.
            Taka sytuacja występuje na granicy rozdziałów 3.2.1 i 3.2.2, kiedy zostaje w zasadzie podjęta decyzja o eliminacji z dalszych rozważań konfiguracji variable-$V_{gs}$, do czego przesłanki dostarcza rozdział 3.2.1.
            Co do rozdziału 3.2.1 to skądinąd nie jest od początku jasne w jakim celu są prowadzone opisywane w nim rozważania.
            Staje się to jasne w rozdziale 3.2.2, kiedy okazuje się, że celem tego rozdziału było rozważenie przesłanek za wyborem jednej z dwóch konfiguracji.
        }
    \end{block}
\end{frame}

\begin{frame}[t]
    \begin{block}{\tb}
        \cit{
            Zdarza się również, że Autorka uznaje, że wniosek w sposób oczywisty wynika z przedstawionych wykresów, co nie jest oczywiste w interdyscyplinarnym środowisku odbiorców.
            Taka sytuacja występuje na granicy rozdziałów 3.3 oraz 3.4.
            Wniosek z rysunku 3.12, który kończy analizę z rozdziału 3.3 jest podsumowany w pierwszym akapicie rozdziału 3.4.
            Takich sytuacji jest więcej, ale nie spotkałem się z oczywistą luką i brakiem informacji, a co najwyżej z jej nieodpowiednią lokalizacją.
        }
    \end{block}
\end{frame}

\begin{frame}[t]
    \begin{block}{\tb}
        \cit{
            Kilkukrotnie Autorka traktuje uzyskane wyniki jako oczywiste i nie tłumaczy która z cech wykresu dowodzi wyciąganego wniosku.
            W kilku przypadkach cechą tą jest różnica nachyleń między dwoma połówkami wykresy -- sytuacja ta dotyczy rys 3.11 oraz 5.18.
        }
    \end{block}
\end{frame}

\begin{frame}[t]
    \begin{block}{\tb}
        \cit{
            Z całkowitych drobiazgów należy zwrócić uwagę na tabelę 3.1, w której zmienna $R_f$ powinna stanowić kolejną kolumnę tabeli.
        }
    \end{block}
\end{frame}

\begin{frame}[t]
    \begin{block}{\tb}
        \cit{
            Co do mankamentów merytorycznych, występujących w dysertacji, jest ich kilka.
            Pierwszy z nich [...] dotyczy faktu, że model źródła, tkanki i sprzężenia jest niejednoznaczny i nie do końca odpowiada rzeczywistości pomiarowej.
            Niezależnie od widma samego źródła, skumulowane efekty pojemności i lokalnego przewodzenia w tkance, nakładają na źródł swoją charakterystykę, która faworyzuje niskie częstotliwości [8].
            Efekty jonowe są składową tego zjawiska [9] ale nie mają dominującego charakteru [10].
        }
    \end{block}
\end{frame}

\begin{frame}[t]
    \begin{block}{\tb}
        \cit{
            Potencjał stały jest przede wszystki efektem brzegowym, związanym z tworzeniem warstwy podwójnej, które z kolei wynika z różnicy potencjałów chemicznych między kontaktującymi się fazami [10].
            Oczywiście rozdzielenie ładunku objętościowego również zachodzi [9], ale jest to efekt fizyczny a nie fizykochemiczny.
            Przemiany elektrochemiczne modą zachodzić dopiero gdy przekroczona jest określona energia aktywacji [10].
            W odniesieniu do elektrod stymulujących piszą o tym Merrill i wsp. -- pozycja [113] literatury -- te rozważania można rozszerzyć na elektrody pomiarowe [8-10].
        }
    \end{block}
\end{frame}

\begin{frame}[t]
    \begin{block}{\tb}
        \cit{
            Fluktuacje termincze dotyczą nie tylko rezystancji ale również pojemności -- taki proces jak tworzenie warstwy podwójnej również jest poddany fluktuacjom.
            W związku z tym nie ma potrzeby odwoływania się do rezystancyjnej natury tkanki, po to, żeby uzasadnić użycie twierdzenia Nyquista.
            Należy to raczej uznać za brak modelu.
            Również rezystywny charakter tkanki nerwowej można poddać w wątpliwość -- cytowany przez Autorkę Destexhe ma w swoim dorobku pracę [7] poświęconą temu zagadnieniu.
        }
    \end{block}
\end{frame}

\begin{frame}[t]
    \begin{block}{\tb}
        \cit{
            Czynnikiem, który w sposób zasadniczy wpływa na model błędu i model sprzężenia tkanki z elektrodą jest lokalizacja elektrody referencyjnej.
            Nie ulega wątpliwości, że sonda MEA jest czymś zupełnie innym niż elektroda referencyjna, więc pomiar jest asymetryczny.
            Nie zmienia to jednak faktu, że nadal jest to pomiar bipolarny.
            Niektóre konstrukcje, takie jak opisywany przez Autorkę Neuropixel, są bipolarne i symetryczne (por rys. 2.8, także rys. 2.13) inne nie są (rys. 2.12, 3.7!).
            Zdecydowanie brakuje odniesienia do tej fundamentalnej różnicy.
            Im dalej umieszczona jest elektroda referencyjna tym większy wpływ na sygnał ma interferencja 50 Hz, oraz wszystkie źródła endogenne, w szczególności silny sygnał kardiogenny.
            Przy odległej lokalizacji elektrody odniesienia trudno jest interpretować otrzymane przebiegi jako neurogenne.
        }
    \end{block}
\end{frame}

\begin{frame}[t]
    \begin{block}{\tb}
        \cit{
            Odwrócenie amplitudy obserwowane na elektrodzie 0 wygląda w pierwszym przybliżeniu na wynik zmiany fazy wynikający ze sprzężenia pojemnościowego -- trudno byłoby zinterpretować odwrócenie amplitudy wprost jako odwrócenie kierunku prądu -- jest to jeden z najciekawszych wyników dotyczących samych narzędzi, sugerujący konieczność dalszego rozwoju techniki modelowania.
        }
    \end{block}
\end{frame}

\begin{frame}[t]
    \begin{block}{\tb}
        \cit{
            Odnośnie podnoszonego przez Autorkę faktu zniknięcia podwójnego maksimum widma THD, warto zauważyć, że rozwiązanie takie występowało już jako jeden z wariantów widma w analizie Monte-Carlo (rys 4.9) oraz wykazywało silną zależność od pojemności $C_{gb}$ (rys 3.10), co mogło być przyczyną obserwowanych różnic między widmem zmierzonym a wynikami symulacji.
        }
    \end{block}
\end{frame}

% \begin{frame}[t]
%     \begin{block}{\tb}
%         \cit{

%         }
%     \end{block}
% \end{frame}

\begin{block}{Kluczowe wymagnia}
    \begin{itemize}
        \item optymalizacja szumowa
        \item powierzchnia
        \item pobór mocy
    \end{itemize}
     \end{block}