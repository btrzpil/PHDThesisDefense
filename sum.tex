\begin{frame}{Podsumowanie testów elektronicznych}

    \begin{longtblr}[
        caption = {Parametry przedwzmacniacza na podstawie pomiarów weryfikacyjnych}
        % label = {table:paramResult},
      ]{
        hline{1-2,12} = {-}{0.08em},
      }
      \textbf{Parametr}                                                                 & \textbf{Wartość}                    \\
      Technologia                                                                 & $\SI{180}{\nano\metre}$               \\
      Napięcia zasilania                                                                & $\SI{\pm 1.8}{\volt}$               \\
      Całkowity prąd                                                                    & $\SI{2}{\micro\ampere}$             \\
      Pobór mocy dla pojedynczego kanału                                                & $\SI{7.2}{\micro\watt}$             \\
      Wzmocnienie z zamkniętą pętlą sprzężenia                                          & $\SI{25.9}{\deci\bel}$              \\
      Zakres dolnej częstotliwości granicznej                                           & $\SIrange{0.1}{20}{\hertz}$         \\
      Ekwiwalentny szum wejściowy w zakresie LFP                                        & $\SI{7.5}{\micro\volt_{rms}}$       \\
      Ekwiwalentny szum wejściowy w zakresie AP                                         & $\SI{6.7}{\micro\volt_{rms}}$       \\
      Zniekształcenia harmonioczne THD – $\SI{10}{\milli\volt_{pp}}\ \SI{1.68}{\hertz}$ & $\SI{0.94}{\percent}$               \\
      Powierzchnia pojedynczego przedwzmacniacza                                    & $\SI{60}{\micro\metre}\times\SI{118}{\micro\metre} = \SI{7080}{\micro\metre\squared}$ \\
      \end{longtblr}
    \end{frame}
    
    \begin{frame}{Podsumowanie}
    \begin{itemize}
        \item przeprowadzono analizę nieliniowości wejściowego obwodu sprzęgającego, który jest odpowiedzialny za ustawienie dolnej częstotliwości granicznej;
        \item wykazano, że największe zniekształcenia występują dla częstotliwości sygnałów w okolicy dolnej częstotliwości granicznej;
        \item zaproponowano i zaimplementowano w opracowanym układzie scalonym nowe rozwiązanie dla pseudo-rezystorów stosowanych w obwodzie sprzęgającym;
        \item  opracowany testowy układ scalony zawiera 14 kanałów, każdy kanał został opracowany w ośmiu wersjach umożliwiających weryfikację różnych wariantów projektowych;
        \item testy elektroniczne pokazały, że możliwe jest uzyskanie zadawalających wszystkich parametrów wzmacniacza  przy polu powierzchni  ograniczonym do $\SI{0.0071}{\milli\metre\squared}$ 
        \item przeprowadzone eksperymenty neurobiologiczne potwierdziły, że przy przy pomocy tego układu możemy prowadzić efektywną rejestrację zarówno sygnałów polowych jak również potencjałów czynnościowych.
    \end{itemize}
    \end{frame}

    \begin{frame}{Podziękowania}
      
      \begin{columns}

        \column{.47\textwidth}
        \begin{itemize}
          \item prof. dr hab. inż. Władysław Dąbrowski
          \item dr inż. Paweł Hottowy
          \item dr hab. Ewa Kublik
          \item dr inż. Piotr Wiącek
          \item dr inż. Tomasz Fiutowski
          \item dr inż. Paweł Jurgielewicz
          \item ...
        \end{itemize}
    
        \column{.47\textwidth}
        \begin{itemize}
          \item Interdyscyplinarne Środowiskowe Studia Doktoranckie Fizyczne, Chemiczne i Biofizyczne Podstawy Nowoczesnych
          Technologii i Inżynierii Materiałowej (FCB)
          \item NCN SYMFONIA 1:2013/08/W/NZ4/00691
        \end{itemize}
    \end{columns}
    \end{frame}