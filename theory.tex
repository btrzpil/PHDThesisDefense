% \begin{frame}{}
%     \begin{columns}
%         \column{.48\textwidth}


%         \column{.48\textwidth}
%         \begin{figure}[H]
%             \centering
%             \includegraphics[scale=1.0]{ch1/lfp_ap_spectrum}  
%             \end{figure}	
%     \end{columns}


\begin{frame}{Zakresy amplitud i częstotliwości sygnałów neuronowych}
    \begin{columns}
        \column{.48\textwidth}
        \begin{figure}[H]
            \includegraphics[scale=0.2]{ch1/brain.jpg}
          \end{figure}

        \column{.48\textwidth}
        \begin{figure}[H]
            \centering
            \includegraphics[scale=1.0]{ch1/lfp_ap_spectrum}  
            \end{figure}	
    \end{columns}
    
\end{frame}



\begin{frame}{Schemat typowego kanału rejestracji neuronowej z wykorzystaniem elektrod zewnątrzkomórkowych}

    \begin{columns}
        \column{.48\textwidth}
        \begin{figure}[H]
            \centering
            \includegraphics[scale=0.25]{ch2/chemNeuroInterface.png} 
        \end{figure}
        \column{.48\textwidth}
        % Schemat typowego kanału rejestracji neuronowej i modelu elektrycznego interfejsu tkanka-mikroelektroda: 
        % $Z_{CPA}$ -- element o stałej fazie, 
        % $R_{CT}$ -- rezystancja dla przepływającego prądu przez elektrodę,  
        % $R_{SP}$ -- rezystancja rozproszona elektrolitu, 
        % $V_{HC}$ -- potencjał w interfejsie elektroda -- tkanka. 
 
    \end{columns}
\end{frame}

\begin{frame}{Wymagania stawiane interfejsom neuroelektronicznym umożliwiającym rejestrację sygnałów LFP i AP}

    
\end{frame}

\begin{frame}{Sprzężenie zmiennoprądowe}
    \begin{columns}
        \column{.48\textwidth}

    \begin{figure}[H]
        \centering
        \includegraphics[scale=1.0]{ch2/conceptAC_Harrison.pdf} 
    \end{figure}
    \column{.48\textwidth}
    \begin{alertblock}{Wyzwania zwiazane z sprzęzeniem AC}
pojemności rezystancja wzmocnienie
    \end{alertblock}
    \begin{exampleblock}{Zalety}

    \end{exampleblock}
\end{columns}

\end{frame}



\begin{frame}{Sprzężenie stałoprądowe}
    \begin{columns}
        \column{.48\textwidth}
        \begin{figure}[H]
            \centering
            \includegraphics[scale = 0.7]{ch2/dc_coupling.pdf}
        \end{figure}

    \column{.48\textwidth}
    \begin{alertblock}{Wyzwania zwiazane z sprzęzeniem DC}
saturacja
    \end{alertblock}
    \begin{exampleblock}{Zalety}

    \end{exampleblock}
\end{columns}

\end{frame}


